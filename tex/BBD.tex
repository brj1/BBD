\documentclass{article}

\usepackage[margin=2cm]{geometry}
\usepackage[doublespacing]{setspace}
\usepackage{amsmath}

%\usepackage{times}
%\usepackage{lineno}
%\usepackage[round]{natbib}
%\makeatletter 
%\makeatother

\usepackage{csquotes}
\usepackage[bibstyle=authoryear,citestyle=authoryear,backend=bibtex,sorting=nyt]{biblatex}
\addbibresource{BBD}

\usepackage[labelfont={bf}]{caption}
\usepackage[labelfont=Large,labelformat=simple,justification=raggedright,singlelinecheck=off]{subcaption}

%\usepackage[super]{cite}

\usepackage{my-tocloft}

\def\cftloftitlefont{}
\def\listfigurename{\section*{Figure legends}}
\def\cftfigfont{Figure }
\cftpagenumbersoff{figure}

\usepackage{graphicx}

\graphicspath{{./figs/}{./}}

\renewcommand{\refname}{\section*{References}}

\usepackage{url}
%\urlstyle{same}

%\usepackage[small,compact]{titlesec}

%\titleformat{\section} {\vspace{24pt}\bf\sffamily\MakeUppercase}{\thesection} {0pt} {}
%\titleformat{\section} {\vspace{12pt}\bf\large}{\thesection}{0pt}{}
%\titleformat{\subsection} {\vspace{6pt}\bf}{\thesubsection} {0pt} {\vspace{-2pt}}
%\titleformat{\subsubsection} [runin] {\bf}{\thesubsubsection} {12pt} {}

\usepackage{comment}
\usepackage{multirow}
%\usepackage{pdflscape}
\usepackage{authblk}

\usepackage{xr}

\usepackage[capitalise,nameinlink,noabbrev]{cleveref}
\usepackage{newfloat}

\DeclareFloatingEnvironment[name={Extended Data Figure}]{exfigure}
\crefname{exfigure}{Extended Data Figure}{Extended Data Figures}

\DeclareFloatingEnvironment[name={Extended Data Table}]{extable}
\crefname{extable}{Extended Data Table}{Extended Data Tables}

\crefname{smfigure}{Supplementary Figure}{Supplementary Figures}
\crefname{smtable}{Supplementary Table}{Supplementary Tables}
\crefname{smsection}{Supplementary Materials Section}{Supplementary Information Sections}
\crefname{mysubsmfigure}{Supplementary Figure}{Supplementary Figures}



\newcommand{\code}[1]{\emph{#1}}
\newcommand{\citep}[1]{{\bfseries{(\cite{#1})}}}

% subfigure labels
\newcounter{mysubfigure}[figure]
\newcounter{mysubfiguresmall}[figure]
\newcommand{\sublabel}[1]{\refstepcounter{mysubfigure}\label{#1}\refstepcounter{mysubfiguresmall}\label{#1small}}
\renewcommand{\themysubfigure}{\thefigure \Alph{mysubfigure}}
\renewcommand{\themysubfiguresmall}{\alph{mysubfiguresmall}}
\newcommand{\mysubref}[1]{\ref{#1small}}
\crefname{mysubfigure}{Figure}{Figures}
\crefname{mysubfirguresmall}{}{}
 
\newcommand{\sm}{Supplementary Materials}
\newenvironment{nattable}[1]{\newcommand{\nl}{\\}\newcommand{\fl}{\\\hline}\centering\tabular{*{#1}{l}}}{\endtabular}
\newcommand{\figcaption}[2]{\caption{{\bfseries #1} #2}}
\newcommand{\na}{~}
\newcommand{\tabcaption}[2]{\caption{{\bfseries #1} #2}}

\newcommand{\intro}[1]{\textbf{#1}\vspace{\baselineskip}}

\title{BBD: reconstructing latent HIV integration dates in BEAST 2}
%Short title: BBD: blind dating in BEAST

\author[1,2]{Bradley R. Jones}
\author[1,2,*]{Jeffrey B. Joy}
\affil[1]{BC Centre for Excellence in HIV/AIDS, 608 --- 1081 Burrard Street, Vancouver, V6Z 1Y6, Canada}
\affil[2]{Department of Medicine, University of British Columbia, 2775 Laurel Street, Vancouver, V5Z 1M9, Canada}
\affil[*]{To whom correspondence should be addressed: BC Centre for Excellence in HIV/AIDS, 608 --- 1081 Burrard Street, Vancouver, V6Z 1Y6, Canada; jjoy@cfenet.ubc.ca}
\date{}

\begin{document}
	\maketitle
	
	\begin{abstract}
	
	\end{abstract}
	Keywords: human immunodeficiency virus 1, viral latency, reservoirs, Bayesian analysis, dating, birth-death models
	
	\section*{Introduction}
		Combined antiretroviral therapy (cART) can reduce viremia levels of humman immunodeficiency virus (HIV) in patients down to undetectable levels reducing the progression of aquired immune deficiency syndrome (AIDS) and transmissity to near zero levels. However cART does not offer a cure to HIV infection due to the presence of the HIV latent reservoir. During active infection HIV virions integrate their viral genome into the host's T-cells. Most of these cells go on to produce virus and then expire from cystosis within a day or two. However a small proportion of infected cells do not  and can persist for years and decades through clonal expansion. These latent cells can be reactivated by external forces to produce virus.
		
		Previous studies have employed genetic similarity \citep{Zanini15,Brodin16} or linear regression \citep{blinddating} with maximum likelihood phylogenetic trees to estimate the integration dates of reservoir sequences. However a more statistically sound approach would use Bayesian methods. Here we present a Java module for BEAST 2 \citep{beast2} to do just this.
	
	\section*{Methods}
		\subsection{BEAST blind dating model}
	
		\subsection{BBD module}
			A Java module for BEAST 2.4 was created which implements our model by extending the MRCAPrior and BirthDeathSkylineModel classes of BEAST 2 \citep{beast2} and BDSKY \citep{bdsky} respectively. The module is avaliable for download at available at https://github.com/brj1/BBD.
		
		\subsection{Simulated data}
		
		\subsection{Curated data}
			We utilized patient HIV sequence data from X individuals (participants 1--Y) from \cite{blinddating} (GenBank: MG822917--MG823179).
		
	\section*{Results}
	
	
	\section*{Discussion}
	
	
	\printbibliography{}
	
	\section*{Appendix}
		\subsection{Tree Likelihood}
			The tree likelihood used by BBD was extrapolated from the likelihood used in the the BDSKY module for BEAST2, which is presented in \cite{bdsky}{}.
		
			Given $R$: reproductive number, $\delta$: removal rate, $s$: sampling proportion, $t$: sampling times.
			
			Define $\lambda = R \delta$, $\mu =(1 - s) \delta$, $\psi = s \delta$, $p_{m+1} = 1$,
			\begin{align*}
				A &= \sqrt{(\lambda - \mu - \psi)^2 + 4 \lambda \psi}, \\
				B_i &= \frac{(1-2(1-\rho_i)p_{i+1})\lambda_i + \psi_i}{A_i}, \\
				p_i &= \frac{\lambda_i + \mu_i + \psi_i - A_i \frac{(1 + B_i) - (1 - B_i) e^{A_i(t_{i-1}-t_i)}}{(1 + B_i) + (1 - B_i) e^{A_i(t_{i-1}-t_i)}}}{2 \lambda_i}, \\
				q(T) &= \ln{4} + A (T - t) - 2 \ln{\left((1 + B) + (1 - B) e^{A (T - t)}\right)}, \\
				\ln f(\mathcal{T} | \lambda, \mu, \psi, \rho, t, S) &= q_1(0) + \sum_{j=1}^N\left(\ln \lambda_{l(x_j)} + q_{l(x_j)}(x_j)\right) + \sum_{j=1}^n\left(\ln \psi_{l(y_j)} - q_{l(y_j)}(y_j)\right) + \\ &\qquad{}\sum_{i=1}^m \left(N_i \ln \rho_i + n_i (\ln{(1-\rho_{i-1})} + q_{i}(t_{i-1}))) \right).
			\end{align*}
			
\end{document}